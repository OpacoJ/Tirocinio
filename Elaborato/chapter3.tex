\chapter{Programmazione Parallela}
Cominciamo a discutere della parallelizzazione in senso più concreto partendo dalla \textbf{prima legge di Moore}:\\\\
\textit{"La complessità di un microcircuito, misurata ad esempio tramite il numero di transistori per chip, raddoppia ogni 18 mesi"}\\\\
Dal punto di vista industriale la creazione di processori più potenti aventi una frequenza di clock superiore comincia a perdere significato, come dimostrato dal fatto che la potenza dei singoli core abbiano avuto un improvviso blocco verso gli inizi degli anni 2000. Questo fenomeno ha reso evidenti i limiti della legge di Moore sopracitata: di per sè i processori non sono migliorati, eppure la potenza di calcolo delle architetture non ha accennato a rallentare. Come ci spieghiamo tutto ciò?\\
Di certo uno sviluppo tecnologico c'è stato, si è solo spostato verso un altro campo: i multiprocessori. Raggiunti i limiti fisici legati alla capacità di elaborazione dei core, la strategia migliore (e per certi versi anche la sola disponibile) per aumentare la potenza di calcolo è stata la concentrazione di molteplici processori installati sulla stessa macchina che lavorano in parallelo.\\
A causa di ciò, il problema della parallelizzazione è concetto sempre più presente nel panorama informatico moderno, può aprire svariate opportunità di ricerca anche dal punto di vista dei linguaggi di programmazione. L'idea di concedere ad un programmatore gli strumenti giusti per ottimizzare le risorse della macchina già nella compilazione del codice da eseguire è al tempo stesso affascinante e allettante. Non è quindi un caso che numerosi linguaggi si siano attrezzati per fornire strutture adatte a questo scopo.
\newpage
\noindent
In generale, quando si tratta di paradigmi di programmazione, distinguiamo due diverse tecniche di programmazione parallela:
\begin{itemize}
\item{implicita: il sistema riconosce autonomamente i meccanismi da adoperare per dividere il problema in modo da poterne eseguire le parti parallelamente; di fatto il programmatore non effettua nessuna precisazione sulla natura dell'esecuzione e scrive il codice come se fosse un programma sequenziale;}
\item{esplicita: il ruolo del programmatore è quello di partizionare il problema nel modo da lui ritenuto migliore; la stesura del codice è cruciale.}
\end{itemize}
La parallelizzazione esplicita è quella che ci interessa maggiormente: vogliamo trovare un modo efficace per parallelizzare un'esecuzione in maniera diretta, definendo le parti da suddividere all'interno del problema già a livello di codice; ovviamente il vincolo richiesto è di ottenere in parallelo lo stesso risultato che si otterrebbe nell'esecuzione sequenziale (in questo caso si parla anche di \textit{parallelismo deterministico}). 

%\chapter{Linguaggi di Programmazione Paralleli}

%\chapter{Python}

%\chapter{Chapel (CRAY)}

\section{Haskell}
Tra tutti i linguaggi che consentono una gestione delle risorse interne delle architetture multiprocessore, Haskell è un ottimo candidato per essere il più fruibile tra tutti quelli disponibili.\\
Oltre a fornire un'interessante serie di caratteristiche formali, quali la programmazione funzionale, la lazy evaluation, le funzioni higher-order e altre che vedremo in dettaglio, possiede già un ricco quantitativo di librerie e strutture che consentono la creazione di programmi paralleli già nel codice.
\subsection{Introduzione al linguaggio Haskell}
Haskell è un linguaggio funzionale puro creato alla fine degli anni '80, disponibile e integrabile su tutte le principali piattaforme software odierne. Possiede le caratteristiche più note dei linguaggi di programmazione (funzionali e non) tra cui la lazy evaluation, il polimorfismo, le funzioni di ordine superiore e supporta il lambda calcolo, su cui si regge dal punto di vista matematico. Caratterizzato da una tipizzazione forte e statica, supporta i tipi di dati più comuni, come gli interi e i valori booleani, e permette di crearne di nuovi sotto forma di strutture dati.\\
Particolare di non poco conto per il nostro interesse, è già pienamente avviato nel mondo dell'elaborazione parallela e dispone di numerose librerie in open source che supportano programmazione parallela e concorrente.\\
Descrivere nel dettaglio un intero linguaggio di programmazione rischia di risultare troppo complicato, pertanto ci limiteremo a descriverne i punti salienti per concentrarci sul lato della programmazione parallela; un tutorial su come muovere i primi passi in Haskell è disponibile al sito \url{https://www.schoolofhaskell.com/}.
\subsection{Installazione}
Per poter eseguire correttamente Haskell è necessario installare il compilatore associato, GHC, the Glasgow Haskell Compiler, disponibile online su \url{https://www.haskell.org/ghc/} per tutte piattaforme Linux, Windows e Mac OS.\\
Oltre a questo è disponibile anche il tool Cabal per l'installazione di package aggiuntivi, scaricabile dal sito \url{https://www.haskell.org/cabal/}.\\
Il package utilizzato per questo lavoro sarà Parallel (\url{https://hackage.haskell.org/package/parallel}) per l'esecuzione parallela.\\
Inoltre è consigliabile un editor di programmazione adatto allo scopo. Un semplice editor di testo è sufficiente, ma è consigliabile utilizzare IDE più specifici, come Leksah o Eclipse. I file di haskell hanno tutti estensione \textit{.hs} .
\subsection{Perché Haskell?}
Ci sono svariati motivi legati all'utilizzo di Haskell per programmare in parallelo; nello specifico:
\begin{itemize}
\item{il paradigma sfruttato da Haskell facilita la creazione di codice pulito e conciso;}
\item{possiede una semantica relativamente semplice da acquisire;}
\item{il codice conciso permette di evitare errori di programmazione o di rilevarne più facilmente;}
\item{programmi molto facili da testare;}
\item{consente un livello di astrazione molto alto;}
\item{le librerie disponibili consentono già una programmazione parallela implicita ad alto livello.}
\end{itemize}
Va fatto notare che la maggior parte delle caratteristiche che sono state enunciate sono garantite dal semplice fatto che Haskell è un linguaggio funzionale. In generale i linguaggi funzionali hanno sempre dimostrato un ottimo utilizzo nella progettazione del software, poiché, consentono una maggiore fruibilità di altre fasi di lavorazione, quali stesura delle specifiche (test funzionali) e manutenzione del software (software inspection facilitato dal codice conciso).\\
\subsection{Linguaggio funzionale: un linguaggio matematicamente puro}
Si pensi ai più famosi, come C, Java e così via. Rappresentano la tecnica di programmazione più classica e "antica": ogni programma è composto da una serie di istruzioni eseguite sequenzialmente, una dopo l'altra. Le varie funzioni (in C) e metodi (in Java) che compongono i programmi sono di fatto \textit{procedure} che lavorano in maniera sequenziale.\\
Al contrario, Haskell è un linguaggio \textit{funzionale}; appartiene, cioè, ad una categoria di linguaggi  in cui ogni elemento che compone il programma è da funzioni matematiche pure, da cui il nome del paradigma. Una funzione in un linguaggio funzionale non esegue calcoli come nella sua controparte imperativa, ma, come in matematica, mappa elementi di un insieme (Dominio) in un altro (Codominio).\\
In questo caso si dice che un elemento $y \in \textit{Y}$ dipende da $x \in \textit{X}$ in base alla relazione:
\begin{center}
$f:X \to Y$
\end{center}
Ad ogni elemento \textit{x}, la funzione \textit{f} associa univocamente il valore \textit{y} nel Codominio \textit{Y}:
\begin{center}
$f(x) = y$
\end{center}
Nel caso di Haskell, la questione è molto simile, anche dal punto di vista semantico. Osserviamo un esempio semplice, come la funzione incremento:
\begin{verbatim}
module Esempio where

inc :: Int -> Int 
inc x = x + 1
\end{verbatim}
Analizziamo ora il codice riga per riga. In Haskell un programma è sostanzialmente una composizione di moduli (definiti appunto \textit{modules}), una struttura a cui è associato un nome (\textit{Esempio}) e definita da un insieme di valori, tipi di dati e funzioni (in questo caso solo \textit{inc}). Per utilizzare uno specifico modulo all'interno di un programma è sufficiente importarne il contenuto, mediante il comando \texttt{import}.\\
In questo caso osserviamo che \textit{inc} viene prima definito dal punto di vista relazionale: è una funzione da elementi di tipo \textit{Int} a elementi di tipo \textit{Int}.\\
Il valore restituito da \textit{inc} dipenderà dall'input (intero) in ingresso. In particolare (aperta la console ghci) otterremo:
\begin{verbatim}
Prelude> import Esempio
Prelude Esempio> inc 2
3
Prelude Esempio> inc 4
5
Prelude Esempio> inc 100
101
\end{verbatim}
Ovviamente, poiché abbiamo limitato la funzione a parametri interi, l'esecuzione di \textit{inc} su valori in floating point solleverà un'eccezione (tipizzazione statica di Haskell).
\subsubsection{Funzioni di ordine superiore}
Ci addentriamo ora in uno dei punti chiave dei linguaggi funzionali. Come già specificato, in questo tipo di linguaggi ogni elemento che compone il programma è rappresentato da una funzione matematica. Ciò che rende effettivamente potenti i linguaggi funzionali è lo sfruttamento di questa caratteristica per creare quelle che vengono definite funzioni \textbf{higher-order} o di ordine superiore.\\
Una funzione higher-order è un particolare tipo di funzione che accetta altre funzioni come parametri o ne restituisce altre come risultato.\\
Facciamo un esempio aggiungendo al codice di prima una nuova funzione:
\begin{verbatim}

myOperator :: (Int -> Int) -> Int -> Int
myOperator f n = (f n) * 2 

\end{verbatim}
In questo caso la funzione \textit{myOperator} accetta in ingresso una funzione da \textit{Int} a \textit{Int} e un valore \textit{Int}. Il suo valore equivale al valore della funzione operata sull'intero in input moltiplicato per 2. Poiché la funzione \textit{inc} corrisponde alle richieste del primo parametro, possiamo sfruttarla per fare qualche esempio di esecuzione:
\begin{verbatim}
Prelude Esempio> myOperator inc 2
6
Prelude Esempio> myOperator inc 4
10
Prelude Esempio>  myOperator inc 100
202
\end{verbatim}
Questi sono esempi basilari, ma dovrebbero rendere conto delle potenzialità inerenti ad un linguaggio funzionale come Haskell. Le funzioni di ordine superiore non solo consentono un alto livello di modularità del codice, ma aprono le porte a numerose funzionalità, tra cui la capacità di incapsulare varie funzioni per variarne meccanismi e risultati.
\subsubsection{Astrazione}
La capacità di creare funzioni di ordine superiore è uno strumento potente in Haskell, soprattutto se associato all'alto livello di astrazione di cui si compone il linguaggio Haskell. Si definisce astrazione la capacità di creare elementi il cui funzionamento interno è definito in maniera implicita e non diretta. Pensiamo alla funzione \textit{myOperator} di prima: pur necessitando di un parametro avente una particolare struttura (\textit{Int -> Int}) non è stato necessario fare alcuna assunzione sul calcolo che essa eseguiva. Potevamo creare qualsiasi funzione che rispettasse le caratteristiche richieste e il programma avrebbe agito di conseguenza in fase di esecuzione.\\
L'astrazione consente di modulare il codice a vantaggio della progettazione generale e permette la creazione di strutture ad alto livello intercambiabili e di semplice utilizzo. Un buon livello di astrazione è generalmente indice di una buona programmazione, facilita la progettazione software e, nel nostro caso, consente di implementare al meglio le funzioni che gestiscono il parallelismo.
\subsubsection{Monade}
Altro elemento cardine di Haskell è la \textbf{Monade}. Dal punto di vista matematico, preso un insieme \textit{X} di definizione, una monade è una \textit{un monoide di X nella categoria degli endofuntori di X}(def. Mac Lane\cite{maclane}). Formalmente definiamo una monade come una tripla \texttt{< M, return, >{}>= >}, in cui:
\begin{itemize}
\item{\texttt{M} rappresenta la tipologia del costrutto;}
\item{\texttt{return} è una funzione che definisce una monade a partire da suo contenuto;}
\item{\texttt{>{}>=} è l'operazione di binding; permette la concatenazione o la composizione di monadi della stessa tipologia.}
\end{itemize}
In informatica le monadi sono essenzialmente strutture che esprimono classi di computazioni concatenabili e risultano particolarmente utili nei linguaggi funzionali. Possiamo identificare una monade come una struttura dati con uno stato  associato. Tipicamente è utilizzata per modellare un aspetto del mondo esterno al programma e   permette   ai   linguaggi funzionali di rimanere puri, cioè senza effetti collaterali.\\
Si tratta comunque di un argomento ampio e complesso, che ci sarà utile quando andremo a definire una tecnica particolare di programmazione parallela; per maggiori informazioni è disponibile una descrizione più dettagliata sul sito della documentazione Haskell: \url{https://wiki.haskell.org/Monad}.
\subsection{Parallelismo in Haskell}
All'inizio del capitolo abbiamo distinto due categorie di parallelismo, implicito ed esplicito, stabilendo di voler concentrare la nostra attenzione sulla seconda. Nelle librerie che utilizzeremo Haskell garantisce una sorta di via di mezzo, una forma definita semi-esplicita: per l'esecuzione parallela sono richiesti dall'utente solo alcuni aspetti della coordinazione tra le varie partizioni del programma. Tuttavia ciò non è assolutamente un fatto negativo: cedendo al sistema la maggior parte del lavoro "sporco", ciò consente di focalizzare l'attenzione solo sulla stesura del codice.\\
Il nostro lavoro sarà solo ed esclusivamente partizionare il problema all'interno del codice in sezioni che possano essere eseguite parallelamente, senza precisazioni sulle caratteristiche della macchina che andrà ad eseguire il codice, nemmeno il numero di processori di cui dispone. Tutte queste caratteritiche andranno definite solo in fase di esecuzione.\\
L'obiettivo è di mantenere i processori in attività sul problema limitando le interazioni tra gli stessi. I problemi derivanti da questa programmazione non mancano. In primis vi è la dipendenza che può intercorrere tra i dati: due sezioni possono condividere una stessa sezione di memoria o essere l'una in attesa di un risultato offerto dall'altra. Ciò porta ad una sorta di ritorno alla sequenzialità non accettabile visto il nostro obiettivo. In secondo luogo vi è il concetto di granularità dei threads, ovvero la dimensione delle partizioni che andremo a creare. Threads troppo piccoli rischiano di diventare un collo di bottiglia per l'esecuzione, che impegna i processori principalmente a coordinarsi piuttosto che a eseguire lavoro utile, vanificando (o anche rendendo obsoleti) gli sforzi della parallelizzazione.
\chapter{Parallel Haskell}
Abbiamo affermato che l'approccio di sfruttare Haskell in parallelo (o come viene comunemente chiamato, Parallel Haskell) è di suddividere il lavoro e di eseguirlo su più processori contemporaneamente.\\
L'effettiva esecuzione su un sistema multiprocessore non è implicita nel codice e deve quindi passare con le corrette impostazioni prima a livello di compilatore e successivamente a livello di esecuzione. Bisogna dunque assicurarsi di aver scaricato l'ultima versione di GHC, o di averlo aggiornato di conseguenza e, una volta terminato il codice, eseguire i seguenti passi:
\begin{itemize}
\item{compilare il codice con l'opzione \texttt{-threaded}; esempio \texttt{ghc -threaded foo.hs -o foo};}
\item{una volta ottenuto il compilato, va mandato in esecuzione con l'opzione \texttt{+RTS -N4}; nello specifico \texttt{RTS} lega l'esecuzione con un sistema Real-Time (per la gestione autonoma di threads, memoria e così via), mentre l'opzione \texttt{-N4} specifica il numero di core da utilizzare, in questo caso quattro (omettere il numero di core da usare consente all'esecuzione di utilizzare tutti quelli presenti nella macchina); esempio \texttt{\.foo +RTS -N4};}
\item{per analizzare anche le statistiche temporali dell'esecuzione, così da confrontare la parallelizzazione con la controparte sequenziale, è anche utile compilare il codice aggiungendo l'opzione \texttt{-rtsopts} ed eseguirlo aggiungendo l'opzione \texttt{-sstderr}; l'esecuzione produrrà un file \textit{stder} contenente le specifiche sul tempo di esecuzione e sulle risorse impiegate.}
\end{itemize}
\section{Parallelismo puro: Control.Parallel}
Cominciamo a descrivere la parallelizzazione esplicita in Haskell partendo dalla libreria più importante in questo ambito, ciò che ci consentirà di trasformare il nostro codice sequenziale in codice parallelo.\\
La libreria Control.Parallel si compone essenzialmente di due funzioni, $par$ e $pseq$, così definite:
\begin{verbatim}
par :: a -> b -> b
pseq :: a -> b -> b
\end{verbatim}
La funzione \texttt{par f1 f2}, esprimibile anche come \texttt{f1 `par` f2}, si traduce sostanzialmente in: "l'esecuzione di f1 può essere parallelizzata con f2; il valore restituito è il risultato di f2". La funzione \textit{par} quindi definisce \textit{f1} come un thread che può essere eseguito in parallelo su un altro processore libero e restituisce il risultato di \textit{f2}.\\
Questa funzione viene usata per effettuare l'esecuzione parallela di due elementi, di cui il valore del primo non è richiesto immediatamente.\\
Sembra una buona soluzione, tuttavia questo non basta per definire il parallelismo in ogni situazione: il fatto che il primo termine possa venire parallelizzato non implica che verrà effettivamente eseguito in parallelo su un altro processore, almeno dal punto di vista del compilatore. Osserviamo il seguente esempio per capirne il motivo:
\begin{verbatim}
parOp = f1 `par` (f1 + f2) where
            f1 = fib 20
            f2 = fib 20
\end{verbatim}
In questo programma abbiamo definito una procedura che restituisce la somma di due funzioni, entrambe definite come l'elemento 20 della serie di Fibonacci (ipotizziamo di aver già creato una funzione che calcoli tale valore, in seguito ne vedremo un'implementazione effettiva). La prima componente di \textit{par} definisce \textit{f1} come thread parallelizzabile, dopodiché calcola $f1 + f2$. La funzione somma in Haskell funziona in modo da calcolare per primo l'elemento sinistro e poiché \textit{f1} viene richiesto per primo\footnote{Ricordiamo che ogni elemento di Haskell viene valutato on demand.} il sistema real time rimane sul processore che ha eseguito la richiesta, ne valuta il risultato e solo dopo aver terminato la valutazione di \textit{f1} calcola \textit{f2} e restituisce il risultato della somma. Di fatto in questo modo abbiamo vanificato gli sforzi per la parallelizzazione: l'esecuzione è rimasta fissa su un unico processore.\\
Una cattiva gestione di questa caratteristica può indurre facilmente a errori concettuali, peraltro non facilmente tracciabili, poiché non sollevano eccezioni (eccezion fatta per altri errori di programmazione, l'esecuzione porterà sempre allo stesso risultato). Una soluzione immediata è di riscrivere il codice specificando che avvenga prima l'esecuzione di \textit{f2}, nel seguente modo:
\begin{verbatim}
parOp2 = f1 `par` (f2 + f1) where
            f1 = fib 20
            f2 = fib 20
\end{verbatim}
Il processore che ha eseguito la richiesta definirà \textit{f1} sempre come un thread parallelizzabile, quindi calcolerà \textit{f2}; un processore libero eseguirà in contemporanea il calcolo di \textit{f1}. In questo caso abbiamo ottenuto il parallelismo richiesto.\\
Spesso però la gestione della parallelizzazione non è così immediata. Per questo motivo subentra la seconda funzione del package. La funzione \textit{pseq f1 f2}, esprimibile anche come \textit{f1 `pseq` f2}, è la controparte parallela della funzione \textit{seq}, della libreria standard di Haskell. Nel caso di \textit{seq} si tratta di una funzione sfruttata per ottimizzare le prestazioni dell'esecuzione, mediante la gestione della lazy evaluation: \textit{f1 `seq` f2} restituisce il valore di \texttt{f2} solo dopo che entrambe le funzioni in input hanno terminato il loro ciclo di esecuzione. In questo modo ci assicuriamo che entrambe le funzioni abbiano terminato la loro esecuzione prima di restituire il risultato di \textit{f2} le due funzioni possono essere eseguite in qualsiasi ordine senza problemi. In generale si dice che la funzione \textit{seq} esegue il suo primo argomento in \textit{weak head normal form}.\\
Analogamente la funzione \texttt{pseq f1 f2} si regge sullo stesso principio di base, ma con una particolarità in più: anziché attendere l'esecuzione di entrambe le funzioni, \textit{pseq} vincola che il calcolo di \textit{f2} avvenga solo dopo l'esecuzione di \textit{(f1 + f2)}.\\
Nell'esempio di prima otteniamo lo stesso risultato parallelo scrivendo:
\begin{verbatim}
parOp3 = f1 `par` (f2 `pseq` (f1 + f2)) where
            f1 = fib 20
            f2 = fib 20
\end{verbatim}
Osserviamo che, mentre \textit{f1} viene sempre definito come parallelizzabile, viene forzata l'esecuzione di \textit{f2} prima che venga effettivamente eseguita la somma. In questo modo il processore che ha eseguito \textit{parOp3} calcola \textit{f2}, mentre \textit{f1} passa in esecuzione ad un altro processore libero.\\
I risultati ottenuti con \textit{parOp2} e \textit{parOp3} sono pressoché identici, in termini prestazionali: entrambi dimezzano il tempo di esecuzione rispetto a \textit{parOp}.
\subsection{Esempio: la serie di Fibonacci}
Proviamo a mettere in pratica il parallelismo esplicito sulla serie di Fibonacci. Ricavare l'\textit{n-esimo} elemento della serie in Haskell diventa:
\begin{verbatim}
fib :: Int -> Int

fib 0 = 1
fib 1 = 1
fib n = (fib (n-1)) + (fib (n-2))
\end{verbatim}
Per ora ignoriamo il fatto che si tratti di un'operazione altamente inefficiente per ottenere il valore richiesto. Parallelizziamo ora il codice usando \textit{pseq} e \textit{par}:
\begin{verbatim}
fibpar :: Int -> Int
fibpar 0 = 1
fibpar 1 = 1
fibpar n = (n1 `par` n2) `pseq` (n1 + n2)
           where
               n1 = fibpar (n-1)
               n2 = fibpar (n-2)
\end{verbatim}
In questo caso, dato un generico intero \texttt{n}, vengono creati due threads (o sparks come vengono chiamati in Parallel Haskell), ognuno associato al calcolo di Fibonacci su \texttt{n-1} e \texttt{n-2}, specificando che possono essere eseguiti parallelamente su due processori liberi (funzione \textit{par})e infine che la loro somma deve attendere che restituiscano un risultato (funzione \textit{seq}).\\
Notiamo che  di per sé gli algoritmi sono concettualmente identici: effettuano la ricorsione due volte per chiamata e uniscono i risultati una volta terminate entrambe le esecuzioni.\\
Confrontiamone ora il tempo di esecuzione effettuato su input 46 ($n = \lfloor \log_2(46) \rfloor + 1 = 6$): il codice ha restituito il risultato dopo 260 secondi utilizzando un solo processore, quello parallelo in 58 secondi utilizzando, sulla stessa macchina, 8 processori. Ricaviamo quindi uno speedup di:
\begin{center}
$S(6,8) \ = \frac{260 sec}{58 sec} \ = \ 4.48 \simeq 450 \%$
\end{center} 
e un'efficienza di:
\begin{center}
$E(6,8) \ = \frac{S(6,8)}{8} \ = \ 0.58 \simeq 58 \%$
\end{center}
\newpage
\section{Le strategie di calcolo}
Un altro modo per parallelizzare l'esecuzione in Haskell è l'utilizzo delle \textbf{strategie di valutazione} o più semplicemente \textbf{strategie}. Si tratta di un meccanismo atto a definire parallelismo deterministico che suddivide il codice separando l'algoritmo in sé dal parallelismo. In questo modo, una volta definito un algoritmo, è possibile parallelizzarlo in modi diversi a seconda della strategia applicata.\\
Sostanzialmente una strategia è una funzione in \textit{Eval}, una monade che prende come argomento un tipo generico e ne restituisce il valore.
\begin{verbatim}
type Strategy a = a -> Eval a
\end{verbatim}
L'idea alla base delle strategie è quella di prendere una struttura dati generica e di calcolarne le componenti mediante un'esecuzione sequenziale o parallela, a seconda di come è stata costruita la strategia.\\
Il vantaggio principale delle strategie consiste principalmente nell'essere componibili: trattando una monade, è possibile combinare sequenzialmente strategie semplici per gestire al meglio algoritmi più complessi (mediante l'operazione di binding).
\subsection{La monade Eval}
La monade Eval è una struttura contenuta nel package Contol.Parallel.Strategies. Vediamo nel dettaglio come la sua composizione:
\begin{verbatim}
data Eval a
instance Monad Eval

rpar :: a -> Eval a
rseq :: a -> Eval a

runEval :: Eval a -> a
\end{verbatim}
Si compone principalmente di due funzioni, \textbf{rpar} e \textbf{rseq}, semanticamente identiche alle sopracitate \textbf{par} e \textbf{pseq} del package Control.Parallel (si veda il paragrafo precedente). Tuttavia presentano differenze sostanziali: \textit{rpar} crea il parallelismo, definendo il suo argomento in ingresso come una funzione da calcolare in parallelo (di fatto è una computazione priva di valutazione e se il suo argomento è già stato calcolato altrove la funzione non ha alcun effetto, vanificando il parallelismo); \textit{rseq}, invece, forza il calcolo della funzione richiesta in (weak head normal form). Va fatto notare che entrambe le funzioni il calcolo è eseguito in weak head normal form.\\
Vi è poi un'altra funzione, \textbf{runEval}, finalizzata ad effettuare la computazione della monade e restituirne il risultato.
\subsection{Usare le strategie}
Avviamoci ora verso l'utilizzo del tipo \textbf{Strategy}. Usare una strategia significa prendere in ingresso una struttura dati, utilizzare le funzioni \textbf{rpar} e \textbf{rseq} per creare parallelismo e restituire il valore del parametro in ingresso.\\
In questo caso le funzioni di Eval possono essere riscritte in funzione delle strategie:
\begin{verbatim}
rpar :: Strategy a
rseq :: Strategy a
\end{verbatim}
Chiariamo il loro utilizzo con un esempio. Immaginiamo di aver creato una funzione \textit{f} e di aver poi deciso di eseguirla sfruttando una strategia \textit{s} creata ad hoc. Ricavare il risultato di f diventa:
\begin{verbatim}
runEval (s f)
\end{verbatim}
Spesso questa scrittura viene sostituita da una funzione più compatta, \textit{using}:
\begin{verbatim}
using :: Strategy a -> a
using s x = runEval (s x)
\end{verbatim}
Possiamo quindi riscrivere la funzione precedente come:
\begin{verbatim}
f `using` s
\end{verbatim}
Un corretto utilizzo delle strategie non è solo un metodo stringente per definire parallelismo. Grazie alla potenza delle astrazioni di Haskell è possibile parallelizzare un algoritmo sequenziale già esistente semplicemente applicandovi una strategia per ottenere il risultato.\\
\subsection{Esempio: la funzione map in parallelo}
Vediamo ora come l'utilizzo delle strategie possa rendere la parallelizzazione concisa ed efficace. La funzione map prende in ingresso due parametri: una funzione (\texttt{f :: a -> b}) e una lista di elementi di tipo \texttt{a}. Il nostro intento è di rendere parallela questa operazione senza modificare l'algoritmo di partenza.\newpage
\noindent
La funzione più semplice finalizzata a questo intento è:
\begin{verbatim}
parMapList :: (a -> b) -> [a] -> [b]
parMapList f ls = map f s `using` parList rseq
\end{verbatim}
Osserviamo nel dettaglio le componenti della funzione:
\begin{itemize}
\item{\texttt{map f s} è l'algoritmo a cui applicare la strategia}
\item{\texttt{parList rseq} è la strategia da sfruttare; \texttt{rseq} l'abbiamo già menzionata, parList, invece, è una strategia già implementata in Control.Parallel.Strategies, che presa una strategia su valori \texttt{'a'} ne restituisce un'altra sulle liste di elementi di tipo \texttt{'a'}; in questo caso prende la strategia \texttt{rseq} e ne restituisce una che applica la suddetta ad ogni elemento di un lista.}
\end{itemize}
Osserviamo alcuni elementi fondamentali: prima di tutto non abbiamo creato nessuna funzione ex novo, abbiamo solo assemblato algoritmi preesistenti e vi abbiamo applicato un meccanismo di calcolo; in secondo luogo abbiamo messo in rilievo che è possibile creare strategie di livello superiore senza particolari assunzioni su tipi e strutture.

