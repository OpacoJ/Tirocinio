\chapter{Problemi implementati}
\section{Ordinamento}
\subsection{QuickSort}
\begin{verbatim}
--Quicksort Sequenziale

quicksort :: Ord a => [a] -> [a]
quicksort [] = []
quicksort (x:xs) = left ++ [x] ++ right where
						left = quicksort [a | a <- xs, a <= x]
						right = quicksort [a | a <- xs, a > x]

--Quicksort Parallelo

parquicksort :: Ord a => [a] -> [a]
parquicksort [] = []
parquicksort (x:xs) = (left `par` right) `pseq` (left ++ [x] ++ right)
						where
							left = parquicksort [a | a <- xs, a <= x]
							right = parquicksort [a | a <- xs, a > x]
\end{verbatim}
\newpage
\subsection{Mergesort}
\begin{verbatim}
--Funzioni ausiliarie

merge :: Ord a => [a] -> [a] -> [a]
merge xs [] = xs
merge [] ys = ys
merge (x:xs) (y:ys)
	| x <= y = x:(merge xs (y:ys))
	| otherwise = y:(merge (x:xs) ys)
	
forceList :: [a] -> ()
forceList [] = ()
forceList (x:xs) = x `pseq` (forceList xs)

--Mergesort Sequenziale

mergesort :: Ord a => [a] -> [a]
mergesort [] = []
mergesort [x] = [x]
mergesort xs = merge (mergesort left) (mergesort right)
				where
					(left, right) = splitAt l xs
					l = div (length xs) 2

--Mergesort Parallelo

parmergesort :: Ord a => [a] -> [a]
parmergesort [] = []
parmergesort [x] = [x]
parmergesort xs = ((forceList left) `par` (forceList right))
                  `pseq` (merge left right)
				        where
                             (left1, right1) = splitAt l xs
                             l = div (length xs) 2
                             left = parmergesort left1
                             right = parmergesort right1
\end{verbatim}
\newpage
\section{Algebra Lineare}
\begin{verbatim}
module Matrix where

import Data.List
import System.Random
import System.IO
import Control.Parallel
import Control.Parallel.Strategies
	
type Vector = [Double]
type Matrix = [Vector]	

makeMat :: Int -> Matrix
makeMat n = replicate n [1.0..t] where t = fromIntegral n

zeroes n = replicate n (replicate n 0.0)

identity :: Int -> Matrix
identity n = identity2 n 0

identity2 :: Int -> Int -> Matrix
identity2 n m 
    | n == m = []
    | otherwise = [((replicate m 0) ++ [1] 
                   ++ (replicate (n-m-1) 0))] ++ (identity2 n (m+1))  

--operazioni sequenziali tra matrici

sumVector :: Vector -> Vector -> Vector
sumVector v1 v2 = [a + b | a <- v1, b <- v2]

subVector :: Vector -> Vector -> Vector
subVector v1 v2 = [a - b | a <- v1, b <- v2]
					  
scalar :: Vector -> Vector -> Double
scalar a b = sum (zipWith (*) a b)


sumMatrix :: Matrix -> Matrix -> Matrix
sumMatrix = zipWith (zipWith (+))

subMatrix :: Matrix -> Matrix -> Matrix
subMatrix = zipWith (zipWith (-))

prodMatrix :: Matrix -> Matrix -> Matrix
prodMatrix m1 m2 = [[scalar a b | b <- column] | a <- m1]
                   where column = transpose m2

powMatrix :: Matrix -> Int -> Matrix
powMatrix m 0 = (identity dim) where dim = length m
powMatrix m 1 = m
powMatrix m n = prodMatrix m (powMatrix m (n-1))
	
	
--operazioni parallele tra matrici
	
sumMatPar :: Matrix -> Matrix -> Matrix
sumMatPar a b = (sumMatrix a b) `using` parList rdeepseq

subMatPar :: Matrix -> Matrix -> Matrix
subMatPar a b = (subMatrix a b) `using` parList rdeepseq
	  
prodMatPar :: Matrix -> Matrix -> Matrix
prodMatPar a b = (prodMatrix a b) `using` parList rdeepseq
\end{verbatim}
\newpage
\begin{verbatim}	  
powMatPar :: Matrix -> Int -> Matrix
powMatPar m 0 = (identity dim) where dim = length m
powMatPar m 1 = m
powMatPar m n = prodMatPar m ris
                    where
                         ris = powMatPar m (n-1)
		  
--determinante

deleteColumn :: Matrix -> Int -> Matrix
deleteColumn [] _ = error "Error input Matrix"
deleteColumn m col = a ++ b where (a, _:b) = splitAt col m
	
deleteElement :: Vector -> Int -> Vector
deleteElement x index = left ++ right
	where (left, _:right) = splitAt index x

deleteRow :: Matrix -> Int -> Matrix
deleteRow [] _ = error "Error input Matrix"
deleteRow m row = [deleteElement x row | x <- m]
						

minor :: Matrix -> Int -> Int -> Matrix
minor [] _ _ = error "Error input Matrix"
minor m row col = deleteRow m1 row where m1 = (deleteColumn m col)

det :: Matrix -> Double
det [] = error "Error input Matrix"
det [[x]] = x
det m = sum [a*s*(det m1) | i <- [0..dim-1], 
            let a = (head m !! i), let m1 = (minor m i 0), let s = (-1)^i] 
                where
                     dim = length m
\end{verbatim}
\newpage
\begin{verbatim}		
--determinante parallelo

		
detList :: Matrix -> Vector
detList [] = error "Error input Matrix"
detList [[x]] = [x]
detList m = [a*s*(det m1) | i <- [0..dim-1],
            let a = (head m !! i), let m1 = (minor m i 0), let s = (-1)^i] 
                  where
                      dim = length m
		
pardet :: Matrix -> Double
pardet m = sum ((detList m) `using` parList rpar)
		
--matrice inversa

invert :: Matrix -> Matrix
invert [[]] = [[]]
invert m = transpose [[s*(det m1)/d | i <- [0..dim-1],
               let m1 = (minor m i j), let s = (-1)^(i+j)] | j <- [0..dim-1]]
                        where
                             dim = length m
                             d = det m

--matrice inversa parallela			
			
parinvert :: Matrix -> Matrix			
parinvert m = (invert m) `using` parList rdeepseq
\end{verbatim}
\newpage
\section{Grafi}
\begin{verbatim}
module Graph where

import Matrix

type Graph = ([Node], [Edge])
type Node = Int
type Edge = (Node, Node)


adjMat :: Graph -> Matrix
adjMat (nodes, []) = zeroes dim
                  where
                        dim = length nodes
adjMat (nodes, e:edges) = setEdge e 1.0 (adjMat (nodes, edges))

setEdge :: Edge -> Double -> Matrix -> Matrix
setEdge _ _ [] = error "Error out of bound"
setEdge (x, y) v m = replacePosition x y v (replacePosition y x v m)

replacePosition :: Int -> Int -> Double -> Matrix -> Matrix                     
replacePosition x y v m = left ++ [new] ++ right
                     where
                        (left, old:right) = splitAt x m
                        (left2, _:right2) = splitAt y old
                        new = left2 ++ [v] ++ right2
\end{verbatim}
\newpage
\begin{verbatim}	
normalize :: Matrix -> Matrix            
normalize m = [[reduce x | x <- y] | y <- m]
               where reduce a 
                     | a==0 = 0
                     | otherwise = 1
                     
fullyConnected :: Matrix -> Bool
fullyConnected [] = True
fullyConnected (x:xs) = (all (/= 0.0) x) && fullyConnected xs

isConnected :: Matrix -> Bool
isConnected m = checkConnection m1 m1 where m1 = normalize m



checkConnection :: Matrix -> Matrix -> Bool
checkConnection m1 m2
      | fullyConnected m1 = True
      | m3 == m1 = False
      | otherwise = checkConnection m3 m2
         where m3 = normalize (prodMatrix m1 m2)
\end{verbatim}