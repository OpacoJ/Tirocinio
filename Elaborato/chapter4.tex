\chapter{Problemi implementati}
\section{Esempi}
\subsection{Modulo Esempi}
\begin{verbatim}
module Esempi where

import Control.Parallel
import Control.Parallel.Strategies

fib :: Int -> Int
fib 0 = 1
fib 1 = 1
fib n = (fib (n-1)) + (fib (n-2))

fibpar :: Int -> Int
fibpar n 
    | n < 10 = fib 10
    | otherwise = (n1 `par` n2) `pseq` (n1 + n2)
           where
               n1 = fibpar (n-1)
               n2 = fibpar (n-2)

parOp = f1 `par` (f1 + f2) where
            f1 = fib 35
            f2 = fib 35
            
parOp2 = f1 `par` (f2 + f1) where
            f1 = fib 35
            f2 = fib 35
	        
parOp3 = f1 `par` (f2 `pseq` (f1 + f2)) where
            f1 = fib 35
            f2 = fib 35
            

parMapList :: (a -> b) -> [a] -> [b]
parMapList f ls = map f ls `using` parList rseq

\end{verbatim}
\subsection{Codice di Testing}
\subsubsection{Test Pair Fibonacci}
\begin{verbatim}
import Control.Parallel
import Control.Parallel.Strategies
import Control.Exception
import Data.Time.Clock
import Text.Printf
import System.Environment
import Utility
import Esempi

main = do
  [n] <- getArgs
  let test = [test1,test2,test3,test4,test5] !! (read n - 1)
  t0 <- getCurrentTime
  r <- evaluate (runEval test)
  printTimeSince t0
  print r
  printTimeSince t0

test1 = return (fib 36,fib 35)

test2 = do
  x <- rpar (fib 36)
  y <- rpar (fib 35)
  return (x,y)

test3 = do
  x <- rpar (fib 36)
  y <- rseq (fib 35)
  return (x,y)

test4 = do
  x <- rpar (fib 36)
  y <- rseq (fib 35)
  rseq x
  return (x,y)

test5 = do
  x <- rpar (fib 36)
  y <- rpar (fib 35)
  rseq x
  rseq y
  return (x,y)
\end{verbatim}
\subsubsection{Test ParSeq}
\begin{verbatim}
import Control.Parallel
import Control.Parallel.Strategies
import Control.Exception
import Data.Time.Clock
import Text.Printf
import System.Environment
import Esempi
import Utility


main = do
  [n] <- getArgs
  let test = [test1,test2,test3] !! (read n - 1)
  t0 <- getCurrentTime
  r <- evaluate (runEval test)
  printTimeSince t0
  print r
  printTimeSince t0


test1 = return parOp
test2 = return parOp2
test3 = return parOp3
\end{verbatim}
\section{Ordinamento}
\subsection{Modulo Sorting}
\begin{verbatim}
module Sorting where

import Control.Parallel
import Control.Parallel.Strategies

--Quicksort Sequenziale

quicksort :: Ord a => [a] -> [a]
quicksort [] = []
quicksort (x:xs) = left ++ [x] ++ right where
                        left = quicksort [a | a <- xs, a <= x]
                        right = quicksort [a | a <- xs, a > x]

--Quicksort Parallelo

parquicksort :: Ord a => [a] -> [a]
parquicksort [] = []
parquicksort (x:xs) = (left `par` right) `pseq` (left ++ [x] ++ right)
                        where
                            left = parquicksort [a | a <- xs, a <= x]
                            right = parquicksort [a | a <- xs, a > x]
                            
--Funzioni ausiliarie

merge :: Ord a => [a] -> [a] -> [a]
merge xs [] = xs
merge [] ys = ys
merge (x:xs) (y:ys)
    | x <= y = x:(merge xs (y:ys))
    | otherwise = y:(merge (x:xs) ys)
    
forceList :: [a] -> ()
forceList [] = ()
forceList (x:xs) = x `pseq` (forceList xs)

--Mergesort Sequenziale

mergesort :: Ord a => [a] -> [a]
mergesort [] = []
mergesort [x] = [x]
mergesort xs = merge (mergesort left) (mergesort right)
                where
                    (left, right) = splitAt l xs
                    l = div (length xs) 2

--Mergesort Parallelo

parmergesort :: Ord a => [a] -> [a]
parmergesort [] = []
parmergesort [x] = [x]
parmergesort xs = ((forceList left) `par` (forceList right))
                  `pseq` (merge left right)
                        where
                             (left1, right1) = splitAt l xs
                             l = div (length xs) 2
                             left = parmergesort left1
                             right = parmergesort right1
\end{verbatim}
\subsection{Codice di Testing}
\subsubsection{Test Sorting}
\begin{verbatim}
import Control.Parallel
import Control.Parallel.Strategies
import Control.Exception
import Data.Time.Clock
import Text.Printf
import System.Environment
import Esempi
import Utility

longls = randomList 10000


main = do
  [n] <- getArgs
  let test = [test1,test2,test3,test4] !! (read n - 1)
  t0 <- getCurrentTime
  r <- evaluate (runEval test)
  printTimeSince t0
  print r
  printTimeSince t0

test1 = return (quicksort longls)
test2 = return (parquicksort longls)
test3 =  return (mergesort longls)
test4 =  return (parmergesort longls)
\end{verbatim}
\section{Algebra Lineare}
\subsection{Modulo Matrix}
\begin{verbatim}
module Matrix where

import Data.List
import System.Random
import System.IO
import Control.Parallel
import Control.Parallel.Strategies
    
type Vector = [Double]
type Matrix = [Vector]    

makeMat :: Int -> Matrix
makeMat n = replicate n [1.0..t] where t = fromIntegral n

zeroes n = replicate n (replicate n 0.0)

identity :: Int -> Matrix
identity n = identity2 n 0

identity2 :: Int -> Int -> Matrix
identity2 n m 
    | n == m = []
    | otherwise = [((replicate m 0) ++ [1] ++ 
                   (replicate (n-m-1) 0))] ++ (identity2 n (m+1))  

--operazioni tra vettore
--prodotto vettore x scalare
smul :: Double -> Vector -> Vector
smul val = map (val*)
--somma
sumVector :: Vector -> Vector -> Vector
sumVector = zipWith (+)

--differenza
subVector :: Vector -> Vector -> Vector
subVector = zipWith (-)

--prodotto scalare                      
scalar :: Vector -> Vector -> Double
scalar a b = sum (zipWith (*) a b)


--operazioni sequenziali tra matrici

--somma 
sumMatrix :: Matrix -> Matrix -> Matrix
sumMatrix = zipWith (zipWith (+))

--differenza
subMatrix :: Matrix -> Matrix -> Matrix
subMatrix = zipWith (zipWith (-))

--prodotto
prodMatrix :: Matrix -> Matrix -> Matrix
prodMatrix m1 m2 = [[scalar a b | b <- column] | a <- m1]
                 where column = transpose m2

--potenza
powMatrix :: Matrix -> Int -> Matrix
powMatrix m 0 = (identity dim) where dim = length m
powMatrix m 1 = m
powMatrix m n = prodMatrix m (powMatrix m (n-1))
    
    
--operazioni parallele tra matrici

--somma    
sumMatPar :: Matrix -> Matrix -> Matrix
sumMatPar a b = (sumMatrix a b) `using` parList rdeepseq

--differenza
subMatPar :: Matrix -> Matrix -> Matrix
subMatPar a b = (subMatrix a b) `using` parList rdeepseq
--prodotto
prodMatPar :: Matrix -> Matrix -> Matrix
prodMatPar a b = (prodMatrix a b) `using` parList rdeepseq
--potenza      
powMatPar :: Matrix -> Int -> Matrix
powMatPar m 0 = (identity dim) where dim = length m
powMatPar m 1 = m
powMatPar m n = prodMatPar m ris
                where
                    ris = powMatPar m (n-1)

                  
--determinante

--metodo di Laplace
det :: Matrix -> Double
det [] = error "Error input Matrix"
det [[x]] = x
det m = sum [a*s*(det m1) | i <- [0..dim-1], 
             let a = (head m !! i), let m1 = (minor m i 0),
             let s = (-1)^i] 
    where
        dim = length m
    
deleteElement :: [a] -> Int -> [a]
deleteElement x index = left ++ right where (left, _:right) = splitAt index x

deleteRow :: Matrix -> Int -> Matrix
deleteRow [] _ = error "Error input Matrix"
deleteRow m row = [deleteElement x row | x <- m]
                        

minor :: Matrix -> Int -> Int -> Matrix
minor [] _ _ = error "Error input Matrix"
minor m row col = deleteRow m1 row where m1 = (deleteElement m col)




--eliminazione gaussiana        
gauss :: Matrix -> Matrix
gauss mat 
    | length mat > 1 = eliminate mat
    | otherwise      = mat

eliminate :: Matrix -> Matrix
eliminate (x:xs) = x : (map (0:) (gauss (map (reduce x) xs)))    


reduce :: Vector -> Vector -> Vector
reduce (l:lr) (r:rr) = subVector rr (smul (aux r l) lr) where
        aux    0 0 = 0
        aux x y = x / y
        
det2 :: Matrix -> Double
det2 [] = error "Error input Matrix"
det2 m = product [m1 !! i !! i | i <- [0..dim-1]]
    where
        dim = length m
        m1 = gauss m

--determinante parallelo

--metodo di Laplace    
detList :: Matrix -> Vector
detList [] = error "Error input Matrix"
detList [[x]] = [x]
detList m = [a*s*(det m1) | i <- [0..dim-1], 
             let a = (head m !! i), let m1 = (minor m i 0), let s = (-1)^i] 
    where
        dim = length m
        

pardet :: Matrix -> Double
pardet m = sum ((detList m) `using` parList rpar)



pfold _ [x] = x
pfold mappend xs  = (ys `par` zs) `pseq` (ys `mappend` zs) where
  len = length xs
  (ys', zs') = splitAt (len `div` 2) xs
  ys = pfold mappend ys'
  zs = pfold mappend zs'
  

pardet2 :: Matrix -> Double
pardet2 m = pfold (*) [m1 !! i !! i | i <- [0..dim-1]]
    where
        dim = length m
        m1 = gauss m
        
        
        
--matrice inversa (determinante NON efficiente)
invert :: Matrix -> Matrix
invert [[]] = [[]]
invert m = transpose [[s*(det m1)/d | i <- [0..dim-1], 
                       let m1 = (minor m i j), 
                       let s = (-1)^(i+j)] | j <- [0..dim-1]]
        where
            dim = length m
            d = det m

--matrice inversa (determinante efficiente)        
invert2 :: Matrix -> Matrix
invert2 [[]] = [[]]
invert2 m = transpose [[s*(det m1)/d | i <- [0..dim-1], 
                        let m1 = (minor m i j), 
                        let s = (-1)^(i+j)] | j <- [0..dim-1]]
        where
            dim = length m
            d = det m
            
            
--matrice inversa parallela    (determinante NON efficiente)            
parinvert :: Matrix -> Matrix            
parinvert m = (invert m) `using` parList rdeepseq


--matrice inversa parallela    (determinante NON efficiente)    
parinvert2 :: Matrix -> Matrix            
parinvert2 m = (invert2 m) `using` parList rdeepseq
\end{verbatim}
\subsection{Codice di Testing}
\subsubsection{Test Matrici}
\begin{verbatim}
import Utility
import Matrix

bigMat = makeMat 100
smallMat = makeMat 8
iden = identity 10

test1 = return (prodMatrix bigMat bigMat)
test2 = return (prodMatPar bigMat bigMat)

test3 = return (powMatrix bigMat 2)
test4 = return (powMatPar bigMat 2)

test5 = return (sumMatrix bigMat bigMat)
test6 = return (sumMatPar bigMat bigMat)

test7 = return (invert iden)
test8 = return (parinvert iden)
test9 = return (parinvert2 iden)

main = do
    [n] <- getArgs
    let test = [test1,test2,test3,test4,test5,
                test6,test7,test8,test9] !! (read n - 1)
    t0 <- getCurrentTime
    r <- evaluate (runEval test)
    print r
    printTimeSince t0
\end{verbatim}
\subsubsection{Test Determinante}
\begin{verbatim}
import Utility
import Matrix

bigMat = makeMat 100
smallMat = makeMat 8
iden = identity 10

test1 = return (det iden)
test2 = return (pardet iden)
test3 = return (det2 bigMat)


main = do
  [n] <- getArgs
  let test = [test1,test2,test3] !! (read n - 1)
  t0 <- getCurrentTime
  r <- evaluate (runEval test)
  printTimeSince t0
  print r
  printTimeSince t0
        
        
\end{verbatim}
\section{Grafi}
\subsection{Modulo Graph}
\begin{verbatim}
module Graph where

import Data.List
import Control.Parallel
import Control.Parallel.Strategies

type Graph = (Int, [Edge])
type Edge = (Int, Int)
type AdjMat = [[Bool]]

--matrice di adiacenza vuota

emptyAdjMat :: Int -> AdjMat
emptyAdjMat dim = emptyAdjMat' dim 0

emptyAdjMat' :: Int -> Int -> AdjMat
emptyAdjMat' n m
            | m >= n = []
            |otherwise = [i == m | i <- [0..n-1]]:(emptyAdjMat' n (m+1))

            
--matrice di adiacenza di un grafo            

makeAdjMat :: Graph -> AdjMat
makeAdjMat (nodes, []) = emptyAdjMat nodes
makeAdjMat (nodes, (x, y):edges) = setEdge (x, y) True
                                    (makeAdjMat (nodes, edges))

setEdge :: (Int, Int) -> Bool -> AdjMat -> AdjMat
setEdge _ _ [] = error "Error in setEdge"
setEdge (x, y) v m = replacePosition x y v (replacePosition y x v m)

replacePosition :: Int -> Int -> a -> [[a]] -> [[a]]                            
replacePosition x y v m = left ++ [new] ++ right
                            where
                                (left, old:right) = splitAt x m
                                (left2, _:right2) = splitAt y old
                                new = left2 ++ [v] ++ right2

--"prodotto" di matrici di adiacenza
                        
prodAdjMat :: AdjMat -> AdjMat -> AdjMat
prodAdjMat m1 m2 = [[orVect a b | b <- column] | a <- m1]
                        where column = transpose m2
    
prodAdjMatPar :: AdjMat -> AdjMat -> AdjMat
prodAdjMatPar a b = (prodAdjMat a b) `using` parList rdeepseq
                        
orVect :: [Bool] -> [Bool] -> Bool
orVect a b = foldOr (zipWith (&&) a b)

foldOr :: [Bool] -> Bool
foldOr [x] = x
foldOr (x:xs) = x || (foldOr xs)

                
--sviluppo n-esimo delle matrici di adiacenza (n scatti)
                        
powAdjMat :: AdjMat -> Int -> AdjMat
powAdjMat m 0 = emptyAdjMat dim where dim = length m
powAdjMat m 1 = m
powAdjMat m n = prodAdjMat m (powAdjMat m (n-1))

powAdjMatPar :: AdjMat -> Int -> AdjMat
powAdjMatPar m 0 = emptyAdjMat dim where dim = length m
powAdjMatPar m 1 = m
powAdjMatPar m n = prodAdjMatPar m (powAdjMatPar m (n-1))

--connettività del grafo
isFullyConnected :: Graph -> Bool
isFullyConnected (n, e) = not(any (elem True) m) where    
                                m = powAdjMat adj_mat n
                                adj_mat = makeAdjMat (n, e)
                                
isFullyConnectedPar :: Graph -> Bool
isFullyConnectedPar (n, e) = not(any (elem True) m) where    
                                m = powAdjMatPar adj_mat n
                                adj_mat = makeAdjMat (n, e)
\end{verbatim}
\subsection{Codice di Testing}
\subsubsection{Test Determinante}
\begin{verbatim}
import Utility
import Graph

g = (10, [(0,1),(0,2),(0,3),(1,2)(1,4),(2,5),(2,8),(4,7),(8,9)])

test1 = return (isFullyConnected g)
test2 = return (isFullyConnected g)


main = do
  [n] <- getArgs
  let test = [test1,test2] !! (read n - 1)
  t0 <- getCurrentTime
  r <- evaluate (runEval test)
  printTimeSince t0
  print r
  printTimeSince t0
        
        
\end{verbatim}
\section{Funzioni ausiliarie}
\subsection{Modulo Utility}
\begin{verbatim}
module Utility where

import Control.Parallel
import Control.Parallel.Strategies
import Control.Exception
import Data.Time.Clock
import Text.Printf
import System.Environment
import System.Random

randomList :: Int -> [Int]
randomList n = take n (randomRs (1, n) (mkStdGen 42))
                 
printTimeSince t0 = do
  t1 <- getCurrentTime
  printf "time: %.2fs\n" (realToFrac (diffUTCTime t1 t0) :: Double)
\end{verbatim}
