\chapter*{Prefazione}
In campo informatico, soprattutto dal punto di vista della programmazione, lo studio degli algoritmi è alla base di qualsiasi tipo di esecuzione. Tuttavia, oltre al caso sequenziale, a cui tutti siamo generalmente abituati quando scriviamo un programma, esiste un'alternativa chiamata "parallelismo": si tratta della capacità di eseguire codice su un sistema dotato di più processori che possono lavorare, appunto, in parallelo.\\
Questo lavoro nasce in seguito ad un'analisi del concetto di algoritmi sia in ambito sequenziale, sia in ambito parallelo, ed alla necessità di trovare un linguaggio consono a questa esigenza. Tra tutti, Haskell ha rappresentato una delle scelte migliori in questo campo. Forte delle sue caratteristiche di linguaggio funzionale e di una vasta gamma di librerie che supportano vari livelli di programmazione parallela, Haskell consente di esprimere parallelismo sia esplicito che implicito già a livello di codice. Il tutto mediante una semantica di comodo utilizzo, a vantaggio del programmatore.\\
L'elaborato presenterà prima un'introduzione del calcolo sequenziale, per poi passare il concetto di calcolo parallelo (dal punto di vista teorico) e concluderà con una presentazione di alcuni modelli messi a disposizione da Haskell. La parte finale mostrerà, inoltre, una serie di algoritmi implementati e testati su una macchina multi-core.